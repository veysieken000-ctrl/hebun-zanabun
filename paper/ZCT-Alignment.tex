\documentclass[11pt]{article}
\usepackage{amsmath}
\usepackage{graphicx}
\usepackage{amssymb}
\usepackage{graphicx}
\usepackage{geometry}
\geometry{margin=1in}

\begin{thebibliography}{9}

\bibitem{amodei2016}
Amodei, D., et al.
\textit{Concrete Problems in AI Safety}.
arXiv:1606.06565 (2016).

\bibitem{misgen}
Langosco, L., et al.
\textit{Goal Misgeneralization in Deep Reinforcement Learning}.
arXiv:2210.01790 (2022).

\bibitem{russell}
Russell, S.
\textit{Human Compatible}.
Viking (2019).

\end{thebibliography}
\title{Upper-Measure Attention as an Alignment Control Parameter}
\author{ZCT Framework}
\date{}

\begin{document}

\maketitle

\begin{abstract}
We introduce a minimal dual-layer decision model in which alignment is governed by an upper-measure attention parameter $\lambda \in [0,1]$.

The model distinguishes between a causal-temporal optimization layer (W4) and an upper-measure value layer (W56).

Drift is defined as:

\[
Drift = 1 - \lambda
\]

This provides a continuous measure of alignment degradation.
\end{abstract}

\section{Introduction}

Alignment failures in intelligent systems often arise not from incorrect optimization,
but from insufficient weighting of higher-order value constraints.

We model this as attention collapse:

\[
\lambda \to 0
\]

\section{Formal Model}
Let:

\[
H(s,u) \text{ be the upper-value alignment measure}
\]
\[
E(s,u) \text{ be the environmental compatibility measure}
\]

Decision rule:

\[
Score_\lambda(s,u) = \lambda H(s,u) + E(s,u)
\]

\section{Theoretical Analysis}

Drift is defined as:

\[
Drift = 1 - \lambda
\]
\subsection{Dynamic Attention Collapse}

To move beyond static scalarization, we model attention as a dynamic quantity:
\[
\lambda_{t+1} = \mathrm{clip}\big(\lambda_t - \eta \, P_t,\, 0,\, 1\big)
\]
where $P_t \ge 0$ denotes training pressure (e.g., incentive to optimize proxy reward),
$\eta > 0$ is a step size, and $\mathrm{clip}$ bounds $\lambda$ to $[0,1]$.

This captures a collapse mechanism: sustained pressure can monotonically decrease $\lambda$,
increasing drift $1-\lambda$ and raising misalignment risk.


\section{Results (Toy MDP)}

We ran a toy simulation sweeping $\lambda \in [0,1]$ and recorded:
(i) proxy reward behavior, (ii) true/upper value behavior, and (iii) safety violations.
As $\lambda$ decreases (i.e., drift increases), safety violations rise and the alignment gap worsens.

\subsection{Distribution Shift}

We evaluate a distribution shift where the correlation between proxy reward $E$ and upper-value $H$
drops from $0.85$ (train) to $0.10$ (test). As $\lambda$ decreases, violations and alignment gap
increase more sharply under the shifted test distribution.

\subsection{Quantitative Alignment Metrics}

To formalize alignment degradation, we define two measurable quantities:

\paragraph{Safety Violation Rate}

Let $V(\lambda)$ denote the expected number of safety violations under policy
induced by attention weight $\lambda$.

\[
V(\lambda) = \mathbb{E}_{s \sim D} [\mathbf{1}_{\text{safety violation}}]
\]

Empirically, we observe:

\[
\frac{dV}{d\lambda} < 0
\]

indicating that increased upper-measure attention reduces safety violations.

\paragraph{Alignment Generalization Gap}

We define the alignment generalization gap:

\[
G(\lambda) = | H_{\text{train}}(\lambda) - H_{\text{test}}(\lambda) |
\]

where $H$ is the upper-value measure.

Under distribution shift, $G(\lambda)$ increases as $\lambda$ decreases.

These results support the interpretation of $\lambda$ as a continuous
alignment control parameter.

\subsection{Theoretical Interpretation within Alignment Literature}

Modern alignment research distinguishes between proxy optimization and
true objective optimization. In standard reinforcement learning,
models optimize a reward proxy $R$, which may diverge from the
intended upper-level value $H$ under distribution shift.

In our formulation, the scoring function

\[
S_\lambda(s,u) = \lambda H(s,u) + (1-\lambda)E(s,u)
\]

explicitly parameterizes this trade-off.

\paragraph{Interpretation of $\lambda$}

We interpret $\lambda$ as an alignment weight controlling the relative
influence of upper-level value $H$ versus proxy estimator $E$.

\begin{itemize}
\item $\lambda = 1$: Fully aligned regime (upper-value dominance)
\item $\lambda = 0$: Pure proxy regime
\item $0 < \lambda < 1$: Mixed regime
\end{itemize}

\paragraph{Alignment Collapse Mechanism}

Under sustained external pressure or internal optimization bias,
$\lambda$ may decrease over time:

\[
\lambda_{t+1} = \lambda_t - \eta \cdot \text{pressure}_t
\]

where $\eta > 0$ is a step size.

Since drift is defined as

\[
\text{Drift} = 1 - \lambda,
\]

monotonic decrease of $\lambda$ induces monotonic increase in drift,
which in turn increases safety violations and alignment gap.

This provides a minimal mathematical model for alignment collapse
under proxy dominance.

Alignment research has identified several recurring failure modes,
including reward hacking, specification gaming, and distributional
shift misalignment. A common structural feature in these failures is
proxy dominance: the learned policy increasingly optimizes a proxy
signal that diverges from the intended objective.

Our formulation differs in that we introduce an explicit,
continuous alignment parameter $\lambda$ controlling the
relative influence of upper-level value $H$ and proxy estimator $E$.

\subsection{Relation to Reward Hacking}

Reward hacking can be interpreted as the regime where

\[
\lambda \rightarrow 0
\]

and the proxy term dominates optimization. In this regime,
the alignment gap

\[
\Delta_{\text{align}} = H^* - H(\pi_\lambda)
\]

increases as drift grows.

\subsection{Relation to Distribution Shift}

Standard distribution shift analysis evaluates performance
degradation between training and test environments.

In our framework, distribution shift acts as an amplifier:
as correlation between $E$ and $H$ decreases,
proxy-dominant policies incur sharply increasing violations.

Thus, distribution shift does not merely reduce performance;
it increases alignment risk nonlinearly when $\lambda$ is small.

\subsection{Minimal Collapse Model}

While many alignment discussions describe failure qualitatively,
our model provides a minimal quantitative collapse dynamic:

\[
\lambda_{t+1} = \lambda_t - \eta \cdot \text{pressure}_t
\]

which induces a predictable increase in drift and safety violations.

This creates a bridge between qualitative alignment discourse
and measurable dynamical behavior.


\begin{figure}[h]
\centering
\includegraphics[width=0.9\linewidth]{figures/shift_violations.png}
\caption{Violations vs $\lambda$ under distribution shift (train vs test).}
\label{fig:shiftviol}
\end{figure}

\begin{figure}[h]
\centering
\includegraphics[width=0.9\linewidth]{figures/shift_gap.png}
\caption{Alignment gap $|E-H|$ vs $\lambda$ under distribution shift (train vs test).}
\label{fig:shiftgap}
\end{figure}

\begin{figure}[h]
\centering
\includegraphics[width=0.9\linewidth]{figures/proxy_vs_true.png}
\caption{Proxy (E) vs True/Upper (H) value among chosen actions as a function of $\lambda$.}
\label{fig:proxytrue}
\end{figure}

\begin{figure}[h]
\centering
\includegraphics[width=0.9\linewidth]{figures/violations.png}
\caption{Safety violations vs $\lambda$ (higher drift corresponds to lower $\lambda$).}
\label{fig:violations}
\end{figure}

As Drift increases:

\begin{itemize}
\item Upper-value influence decreases
\item Short-term optimization dominates
\item Misalignment risk increases
\end{itemize}

\section{Contribution}

This framework:

\begin{itemize}
\item Converts alignment into a continuous parameter
\item Makes value neglect measurable
\item Provides a control interpretation of ethical drift
\end{itemize}

\section{Future Work}

\begin{itemize}
\item Dynamic $\lambda(t)$ models
\item Stability proofs
\item Multi-agent drift interaction
\end{itemize}

\section{Related Work}

Alignment failures such as reward hacking and specification gaming
have been extensively discussed in the AI safety literature.

Amodei et al. (2016) highlight concrete safety problems arising
from imperfect reward specification.
Goal misgeneralization work further shows that even correctly specified
training objectives can lead to unintended goal pursuit under distribution shift.

Our framework does not introduce a new reward structure,
but instead provides a control-parameter interpretation:
misalignment is modeled as attention collapse toward higher-order values,
quantified by $\lambda$.

This perspective reframes alignment degradation
as continuous upper-value neglect rather than binary failure.

\section{Novelty Clarification}

The weighted objective form
\[
Score_\lambda(s,u) = \lambda H(s,u) + E(s,u)
\]
is not claimed to be novel as a mathematical structure.

The contribution of this work lies in:

\begin{itemize}
\item Interpreting misalignment as \textbf{attention collapse} toward upper-order values.
\item Introducing a continuous drift metric: $Drift = 1 - \lambda$.
\item Framing alignment degradation as a controllable parameter rather than a binary failure.
\item Providing a minimal experimental protocol to measure drift effects.
\end{itemize}

Thus, the novelty lies in the control-theoretic interpretation of value neglect,
not in the algebraic form of scalarization.
\section{Conclusion}

We introduced an alignment framing based on upper-measure attention $\lambda$ and drift $1-\lambda$,
and provided a minimal evaluation protocol.
Future work focuses on dynamic $\lambda(t)$ and stronger empirical validation.

\end{document}
