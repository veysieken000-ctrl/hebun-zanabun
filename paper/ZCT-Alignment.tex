\documentclass[11pt]{article}
\usepackage{amsmath}
\usepackage{amssymb}
\usepackage{geometry}
\geometry{margin=1in}

\title{Upper-Measure Attention as an Alignment Control Parameter}
\author{ZCT Framework}
\date{}

\begin{document}

\maketitle

\begin{abstract}
We introduce a minimal dual-layer decision model in which alignment is governed by an upper-measure attention parameter $\lambda \in [0,1]$.

The model distinguishes between a causal-temporal optimization layer (W4) and an upper-measure value layer (W56).

Drift is defined as:

\[
Drift = 1 - \lambda
\]

This provides a continuous measure of alignment degradation.
\end{abstract}

\section{Motivation}

Alignment failures in intelligent systems often arise not from incorrect optimization,
but from insufficient weighting of higher-order value constraints.

We model this as attention collapse:

\[
\lambda \to 0
\]

\section{Core Model}

Let:

\[
H(s,u) \text{ be the upper-value alignment measure}
\]
\[
E(s,u) \text{ be the environmental compatibility measure}
\]

Decision rule:

\[
Score_\lambda(s,u) = \lambda H(s,u) + E(s,u)
\]

\section{Drift}

Drift is defined as:

\[
Drift = 1 - \lambda
\]

As Drift increases:

\begin{itemize}
\item Upper-value influence decreases
\item Short-term optimization dominates
\item Misalignment risk increases
\end{itemize}

\section{Contribution}

This framework:

\begin{itemize}
\item Converts alignment into a continuous parameter
\item Makes value neglect measurable
\item Provides a control interpretation of ethical drift
\end{itemize}

\section{Future Work}

\begin{itemize}
\item Dynamic $\lambda(t)$ models
\item Stability proofs
\item Multi-agent drift interaction
\end{itemize}

\end{document}
