LaTeX
\documentclass[12pt]{article}
\usepackage{amsmath, amssymb, amsthm}
\usepackage{geometry}
\usepackage{hyperref}
\geometry{margin=1in}

\title{Zanabûn Framework: \\ 
A Modal--Symbolic Formalization of a Six-Dimensional Ontological Model}

\author{Veysi yê MALA SAF}
\date{Version 1.2 -- Civilizational Phase}

\begin{document}

\maketitle

\begin{abstract}
This paper presents a modal and symbolic formalization of the Hebûn--Zanabûn framework, 
a six-dimensional ontological-epistemological architecture. 
We argue that expansion beyond four-dimensional positivist models is not 
metaphysical speculation but a structural necessity derived from dimensional incompleteness.
The paper provides modal logic structure, symbolic notation, type theory, 
computational abstraction, and proof sketch.
\end{abstract}

\section{Introduction}

Classical positivist epistemology operates within a four-dimensional explanatory frame:

\[
F_4 = \{D_1, D_2, D_3, D_4\}
\]

where:

\begin{itemize}
\item $D_1$ = Physical
\item $D_2$ = Biological
\item $D_3$ = Cognitive
\item $D_4$ = Structural / Legal
\end{itemize}

However, intentional moral agency cannot be reduced to these layers without contradiction.

\section{Dimensional Expansion}

Define:

\[
D = \{D_1, D_2, D_3, D_4, D_5, D_6\}
\]

where:

\begin{itemize}
\item $D_5$ = Moral--Intentional Layer
\item $D_6$ = Ultimate Judgement / Closure Layer
\end{itemize}

\textbf{Dimensional Incompleteness Thesis:}

\[
\forall S \left( Explains(S, D_1...D_4) \wedge \neg Explains(S, Normativity) \rightarrow Incomplete(S) \right)
\]

\section{Modal Logical Structure}

Let:

\[
\Box = \text{Structural Necessity}
\]
\[
\Diamond = \text{Structural Possibility}
\]

Core axiom:

\[
\Box (Incomplete(F_4) \rightarrow \exists D_5 \land \exists D_6)
\]

Thus, moral and closure dimensions are necessary for coherence.

\section{Symbolic Evaluation Model}

Let:

\[
x \in V
\]

where $V$ is the set of ontological entities.

Define:

\[
H(x) = f(D_1(x), D_2(x), D_3(x), D_4(x), D_5(x), D_6(x))
\]

Reductionist systems compute:

\[
H_4(x) = f(D_1...D_4)
\]

But:

\[
H_4(x) \neq H(x)
\]

Therefore:

\[
\Box (H \text{ requires } D_5 \land D_6)
\]

\section{Type Structure}

\begin{align*}
Entity &::= Physical \mid Biological \mid Cognitive \mid Structural \mid Moral \mid Judgemental \\
Decision &::= Action \rightarrow MoralState \\
MoralState &::= Expansion \mid Contraction \\
Judgement &::= TerminalEvaluation
\end{align*}

\section{Computational Abstraction}

\begin{verbatim}
function Evaluate(entity x):
    layers = ExtractLayers(x)

    if Missing(layers.D5) or Missing(layers.D6):
        return "Dimensional Incompleteness"

    return Integrate(layers.D1..D6)
\end{verbatim}

\section{Proof Sketch}

\begin{enumerate}
\item Intentional agency exists.
\item Intentional agency implies normativity.
\item Normativity implies evaluation beyond physical causality.
\item Evaluation implies moral layer ($D_5$).
\item Moral layer implies closure condition ($D_6$).
\item Therefore, six-dimensional closure is necessary.
\end{enumerate}

\section{Conclusion}

The six-dimensional model is a structural completion of positivist epistemology. 
It does not reject empirical science but embeds it in a layered ontological framework.

Without closure, any system remains ontologically open and incomplete.

\end{document}

