\section*{ZCT – Matematik Modeli}

\subsection*{1. Uzaylar}

S : Durum uzayı (W4 koşulları) \\
U : Eylem uzayı \\

\subsection*{2. Ölçü Fonksiyonları}

H(s,u) : Üst ölçü (W56 referansı) \\
E(s,u) : Ekosistem uyumu \\
G(s,u) : Kısa vadeli çıkar \\

\subsection*{3. Karar Fonksiyonu}

Bir birey, s durumunda u eylemini şu bileşimle değerlendirir:

\[
Score(s,u) = \alpha H(s,u) + \beta E(s,u) + \gamma G(s,u)
\]

\alpha, \beta, \gamma \ge 0

\subsection*{4. Yol Kayması}

Yol kayması:

\[
\gamma \uparrow \quad ve \quad \alpha \downarrow
\]

durumunda ortaya çıkar.

\subsection*{5. Uyum Ölçütü}

\[
Alignment(s,u) = H(s,u) + E(s,u)
\]

Alignment arttıkça çatışma azalır.

section*{ZCT – Minimal Derin Model}

\subsection*{1. Uzaylar}
S : Durum (W4 koşulları) \\
U : Eylem \\

\subsection*{2. Ölçüler}
H(s,u) : Üst ölçü (W56 yönü) \\
E(s,u) : Ekosistem uyumu \\

\subsection*{3. Dikkat (Hesaba Katma)}
W4 baskısı, H'nin hesaba katılma düzeyini azaltabilir.

\[
\lambda \in [0,1]
\]

\(\lambda = 1\): H tam hesaba katılır. \\
\(\lambda = 0\): H ihmal edilir.

\subsection*{4. Karar}
\[
Score_\lambda(s,u) = \lambda H(s,u) + E(s,u)
\]

\subsection*{5. Yol Kayması (tanım)}
Yol kayması, \(\lambda\)'nın düşmesiyle başlar.

\[
\lambda \downarrow \;\Rightarrow\; H \text{ etkisi azalır}
\]


