\section*{ZCT – Minimal Derin Model}

\subsection*{1. Uzaylar}
S : Durum (W4 koşulları) \\
U : Eylem \\

\subsection*{2. Ölçüler}
H(s,u) : Üst ölçü (W56 yönü) \\
E(s,u) : Ekosistem uyumu \\

\subsection*{3. Dikkat (Hesaba Katma)}

Üst ölçünün hesaba katılma düzeyi λ ile temsil edilir.

\[
\lambda \in [0,1]
\]

\lambda = 1 → H tam hesaba katılır \\
\lambda = 0 → H tamamen ihmal edilir

\subsection*{4. Karar}

\[
Score_\lambda(s,u) = \lambda H(s,u) + E(s,u)
\]

\subsection*{6. Teorem Hedefleri}

Bu model, aşağıdaki hedef teoremleri üretmek için tasarlanmıştır:

1. Drift Monotonluğu:
   λ₂ ≤ λ₁ ⇒ Drift(λ₂) ≥ Drift(λ₁)

2. λ=0 İhmal:
   Score₀(s,u) = E(s,u)

3. λ=1 Tam Katılım:
   Score₁(s,u) = H(s,u) + E(s,u)

4. Uyum Hedefi:
   λ ↑ ⇒ Drift ↓
