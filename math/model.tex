\section*{ZCT – Matematik Modeli}

\subsection*{1. Uzaylar}

S : Durum uzayı (W4 koşulları) \\
U : Eylem uzayı \\

\subsection*{2. Ölçü Fonksiyonları}

H(s,u) : Üst ölçü (W56 referansı) \\
E(s,u) : Ekosistem uyumu \\
G(s,u) : Kısa vadeli çıkar \\

\subsection*{3. Karar Fonksiyonu}

Bir birey, s durumunda u eylemini şu bileşimle değerlendirir:

\[
Score(s,u) = \alpha H(s,u) + \beta E(s,u) + \gamma G(s,u)
\]

\alpha, \beta, \gamma \ge 0

\subsection*{4. Yol Kayması}

Yol kayması:

\[
\gamma \uparrow \quad ve \quad \alpha \downarrow
\]

durumunda ortaya çıkar.

\subsection*{5. Uyum Ölçütü}

\[
Alignment(s,u) = H(s,u) + E(s,u)
\]

Alignment arttıkça çatışma azalır.

